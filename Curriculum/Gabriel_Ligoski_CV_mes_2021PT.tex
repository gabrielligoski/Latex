\documentclass{my_cvPT}

\begin{document}
\begin{multicols}{2}[
    \titletext{Gabriel}%
        {Ligoski}%  
        {Desenvolvedor de Software}%
        {\href{mailto:gabriel.ligoski@gmail.com}{gabriel.ligoski@gmail.com}}%
        {+55 61 99611-1781}%
        {\href{https://github.com/gabrielligoski}{gabrielligoski} }%
        {\href{https://www.linkedin.com/in/gabriel-ligoski/}{gabriel-ligoski} }%
]
\end{multicols}
\vspace{15mm}
\section{\faFileText}{APRESENTAÇÃO}
Sou estudante de Bacharel em Ciência da Computação (BCC) 
na Universidade de Brasília (UnB). Tenho paixão por novas tecnologias, estou 
sempre em busca de novos desafios e obtendo novos conhecimentos no caminho.
Minha maior proeza foi fazer um pinball multiplayer em assembly.

\vspace{5mm}

\begin{multicols}{2}
\section{\faPencil}{EXPERIÊNCIA}
    
\work{Estágio FullStack / Maio 2021 - Outubro 2021 }%
    {Vobys}%
    {Trabalhei como desenvolvedor fullstack utilizando React no frontend
     \ e Java Spring como backend no desenvolvimento de uma plataforma de gestão.}%
    {Github, Java, Spring, Docker, React, Javascript, Jhipster}
    
\section{\faList}{HABILIDADES}

\textbf{Linguagens:} Java, Python, C, C\#, JavaScript, HTML, CSS, Assembly

\noindent\textbf{Ferramentas:} React, Tailwind, Material-ui, Spring, Heroku, Tensorflow, Selenium, MySql, Postgresql, NumPy, Pandas, Docker

\noindent\textbf{Outros:} Version Control, Web, Data science and ML, Cloud computing

\section{\faGraduationCap}{EDUCAÇÃO}

\school{Bacharel em Ciência da Computação\\ 2019.2 - 2024.1} %
{University of Brasília(UnB), Brasília. DF} %
{\textbf{IRA(índice de rendimento acadêmico) 4.4638/5.00}}

\school{Inglês} %
{Nível avançado} %
{}

\columnbreak

\section{\faPaintBrush}{PROJETOS PESSOAIS}
\begin{itemize}[noitemsep]
    \item Criei um pinball multiplayer em assembly Risc-V.
    \item Criei um app de atendimento para treinadores de futebol como freelancer utilizando o React Native.
    \item Eu criei um bot de discord em python usando git para fazer o upload do aplicativo para a nuvem no heroku.
    \item Fiz um aplicativo usando Tensorflow e web scraping para criar uma IA que prevê resultados de partidas de e-sports.
    \item Criei um site JSP (Java Server Pages) e banco de dados MySql para gerenciamento de eventos com lista de presença.
    \item Eu subi alguns dos meus aplicativos na nuvem usando Gcloud e Heroku, com Ubuntu e Git respectivamente.
\end{itemize}

\section{\faStar}{CERTIFICAÇÕES}

\begin{itemize}[noitemsep]
    \item Java OCAJP 1Z0-808 - Oracle Certified Associate Java Programmer SE 8 Programmer I
\end{itemize}

\section{\faSoccerBallO}{ATIVIDADES / INTERESSES}

Proativo em aprender coisas diversas e gosto de discuti-las. As atividades físicas favoritas são vôlei e krav maga.


\end{multicols}
\end{document}
